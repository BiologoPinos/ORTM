% Options for packages loaded elsewhere
\PassOptionsToPackage{unicode}{hyperref}
\PassOptionsToPackage{hyphens}{url}
%
\documentclass[
]{article}
\usepackage{amsmath,amssymb}
\usepackage{iftex}
\ifPDFTeX
  \usepackage[T1]{fontenc}
  \usepackage[utf8]{inputenc}
  \usepackage{textcomp} % provide euro and other symbols
\else % if luatex or xetex
  \usepackage{unicode-math} % this also loads fontspec
  \defaultfontfeatures{Scale=MatchLowercase}
  \defaultfontfeatures[\rmfamily]{Ligatures=TeX,Scale=1}
\fi
\usepackage{lmodern}
\ifPDFTeX\else
  % xetex/luatex font selection
\fi
% Use upquote if available, for straight quotes in verbatim environments
\IfFileExists{upquote.sty}{\usepackage{upquote}}{}
\IfFileExists{microtype.sty}{% use microtype if available
  \usepackage[]{microtype}
  \UseMicrotypeSet[protrusion]{basicmath} % disable protrusion for tt fonts
}{}
\makeatletter
\@ifundefined{KOMAClassName}{% if non-KOMA class
  \IfFileExists{parskip.sty}{%
    \usepackage{parskip}
  }{% else
    \setlength{\parindent}{0pt}
    \setlength{\parskip}{6pt plus 2pt minus 1pt}}
}{% if KOMA class
  \KOMAoptions{parskip=half}}
\makeatother
\usepackage{xcolor}
\usepackage[margin=1in]{geometry}
\usepackage{color}
\usepackage{fancyvrb}
\newcommand{\VerbBar}{|}
\newcommand{\VERB}{\Verb[commandchars=\\\{\}]}
\DefineVerbatimEnvironment{Highlighting}{Verbatim}{commandchars=\\\{\}}
% Add ',fontsize=\small' for more characters per line
\usepackage{framed}
\definecolor{shadecolor}{RGB}{248,248,248}
\newenvironment{Shaded}{\begin{snugshade}}{\end{snugshade}}
\newcommand{\AlertTok}[1]{\textcolor[rgb]{0.94,0.16,0.16}{#1}}
\newcommand{\AnnotationTok}[1]{\textcolor[rgb]{0.56,0.35,0.01}{\textbf{\textit{#1}}}}
\newcommand{\AttributeTok}[1]{\textcolor[rgb]{0.13,0.29,0.53}{#1}}
\newcommand{\BaseNTok}[1]{\textcolor[rgb]{0.00,0.00,0.81}{#1}}
\newcommand{\BuiltInTok}[1]{#1}
\newcommand{\CharTok}[1]{\textcolor[rgb]{0.31,0.60,0.02}{#1}}
\newcommand{\CommentTok}[1]{\textcolor[rgb]{0.56,0.35,0.01}{\textit{#1}}}
\newcommand{\CommentVarTok}[1]{\textcolor[rgb]{0.56,0.35,0.01}{\textbf{\textit{#1}}}}
\newcommand{\ConstantTok}[1]{\textcolor[rgb]{0.56,0.35,0.01}{#1}}
\newcommand{\ControlFlowTok}[1]{\textcolor[rgb]{0.13,0.29,0.53}{\textbf{#1}}}
\newcommand{\DataTypeTok}[1]{\textcolor[rgb]{0.13,0.29,0.53}{#1}}
\newcommand{\DecValTok}[1]{\textcolor[rgb]{0.00,0.00,0.81}{#1}}
\newcommand{\DocumentationTok}[1]{\textcolor[rgb]{0.56,0.35,0.01}{\textbf{\textit{#1}}}}
\newcommand{\ErrorTok}[1]{\textcolor[rgb]{0.64,0.00,0.00}{\textbf{#1}}}
\newcommand{\ExtensionTok}[1]{#1}
\newcommand{\FloatTok}[1]{\textcolor[rgb]{0.00,0.00,0.81}{#1}}
\newcommand{\FunctionTok}[1]{\textcolor[rgb]{0.13,0.29,0.53}{\textbf{#1}}}
\newcommand{\ImportTok}[1]{#1}
\newcommand{\InformationTok}[1]{\textcolor[rgb]{0.56,0.35,0.01}{\textbf{\textit{#1}}}}
\newcommand{\KeywordTok}[1]{\textcolor[rgb]{0.13,0.29,0.53}{\textbf{#1}}}
\newcommand{\NormalTok}[1]{#1}
\newcommand{\OperatorTok}[1]{\textcolor[rgb]{0.81,0.36,0.00}{\textbf{#1}}}
\newcommand{\OtherTok}[1]{\textcolor[rgb]{0.56,0.35,0.01}{#1}}
\newcommand{\PreprocessorTok}[1]{\textcolor[rgb]{0.56,0.35,0.01}{\textit{#1}}}
\newcommand{\RegionMarkerTok}[1]{#1}
\newcommand{\SpecialCharTok}[1]{\textcolor[rgb]{0.81,0.36,0.00}{\textbf{#1}}}
\newcommand{\SpecialStringTok}[1]{\textcolor[rgb]{0.31,0.60,0.02}{#1}}
\newcommand{\StringTok}[1]{\textcolor[rgb]{0.31,0.60,0.02}{#1}}
\newcommand{\VariableTok}[1]{\textcolor[rgb]{0.00,0.00,0.00}{#1}}
\newcommand{\VerbatimStringTok}[1]{\textcolor[rgb]{0.31,0.60,0.02}{#1}}
\newcommand{\WarningTok}[1]{\textcolor[rgb]{0.56,0.35,0.01}{\textbf{\textit{#1}}}}
\usepackage{graphicx}
\makeatletter
\def\maxwidth{\ifdim\Gin@nat@width>\linewidth\linewidth\else\Gin@nat@width\fi}
\def\maxheight{\ifdim\Gin@nat@height>\textheight\textheight\else\Gin@nat@height\fi}
\makeatother
% Scale images if necessary, so that they will not overflow the page
% margins by default, and it is still possible to overwrite the defaults
% using explicit options in \includegraphics[width, height, ...]{}
\setkeys{Gin}{width=\maxwidth,height=\maxheight,keepaspectratio}
% Set default figure placement to htbp
\makeatletter
\def\fps@figure{htbp}
\makeatother
\setlength{\emergencystretch}{3em} % prevent overfull lines
\providecommand{\tightlist}{%
  \setlength{\itemsep}{0pt}\setlength{\parskip}{0pt}}
\setcounter{secnumdepth}{-\maxdimen} % remove section numbering
\ifLuaTeX
  \usepackage{selnolig}  % disable illegal ligatures
\fi
\usepackage{bookmark}
\IfFileExists{xurl.sty}{\usepackage{xurl}}{} % add URL line breaks if available
\urlstyle{same}
\hypersetup{
  pdftitle={PISCO\_urchin\_tuffy},
  pdfauthor={Andrés Pinos-Sánchez},
  hidelinks,
  pdfcreator={LaTeX via pandoc}}

\title{PISCO\_urchin\_tuffy}
\author{Andrés Pinos-Sánchez}
\date{2025-05-15}

\begin{document}
\maketitle

\#\#AIM OF THE SCRIPT Obtain the year to tear variability of urchin
recruitment (std) in Oregon

\#\#INPUT RAW DATA

This data is from PISCO surveys posted by MENGE, for intertidal
recruitmentacross the coast O.G. data (look for all ``tuffy''
data-sets):
\url{https://data.piscoweb.org/metacatui/view/doi\%3A10.6085\%2FAA\%2Fpisco_recruitment.1477.1}

\begin{Shaded}
\begin{Highlighting}[]
\CommentTok{\# \# Input data from PISCO, data name corresponds to collection year (not deployment)}
\CommentTok{\# PISCO\_data\_1989 \textless{}{-} read\_csv("Datasets/doi\_10.6085\_AA\_pisco\_recruitment.124.1.csv")  |\textgreater{} mutate(year = 1989, .before = 1)}
\CommentTok{\# PISCO\_data\_1990 \textless{}{-} read\_csv("Datasets/doi\_10.6085\_AA\_pisco\_recruitment.126.1.csv")  |\textgreater{} mutate(year = 1990, .before = 1)}
\CommentTok{\# PISCO\_data\_1991 \textless{}{-} read\_csv("Datasets/doi\_10.6085\_AA\_pisco\_recruitment.128.1.csv")  |\textgreater{} mutate(year = 1991, .before = 1)}
\CommentTok{\# PISCO\_data\_1992 \textless{}{-} read\_csv("Datasets/doi\_10.6085\_AA\_pisco\_recruitment.130.1.data") |\textgreater{} mutate(year = 1992, .before = 1)}
\CommentTok{\# PISCO\_data\_1993 \textless{}{-} read\_csv("Datasets/doi\_10.6085\_AA\_pisco\_recruitment.132.1.data") |\textgreater{} mutate(year = 1993, .before = 1)}
\CommentTok{\# PISCO\_data\_1994 \textless{}{-} read\_csv("Datasets/doi\_10.6085\_AA\_pisco\_recruitment.134.1.csv")  |\textgreater{} mutate(year = 1994, .before = 1)}
\CommentTok{\# PISCO\_data\_1995 \textless{}{-} read\_csv("Datasets/doi\_10.6085\_AA\_pisco\_recruitment.136.1.data") |\textgreater{} mutate(year = 1995, .before = 1)}
\CommentTok{\# PISCO\_data\_1996 \textless{}{-} read\_csv("Datasets/doi\_10.6085\_AA\_pisco\_recruitment.138.1.data") |\textgreater{} mutate(year = 1996, .before = 1)}
\CommentTok{\# \# 1997 skipped}
\CommentTok{\# PISCO\_data\_1998 \textless{}{-} read\_csv("Datasets/doi\_10.6085\_AA\_pisco\_recruitment.140.1.data") |\textgreater{} mutate(year = 1998, .before = 1)}
\CommentTok{\# PISCO\_data\_1999 \textless{}{-} read\_csv("Datasets/doi\_10.6085\_AA\_pisco\_recruitment.142.1.data") |\textgreater{} mutate(year = 1999, .before = 1)}
\CommentTok{\# PISCO\_data\_2000 \textless{}{-} read\_csv("Datasets/doi\_10.6085\_AA\_pisco\_recruitment.144.1.csv")  |\textgreater{} mutate(year = 2000, .before = 1)}
\CommentTok{\# PISCO\_data\_2001 \textless{}{-} read\_csv("Datasets/doi\_10.6085\_AA\_pisco\_recruitment.146.1.data") |\textgreater{} mutate(year = 2001, .before = 1)}
\CommentTok{\# PISCO\_data\_2002 \textless{}{-} read\_csv("Datasets/doi\_10.6085\_AA\_pisco\_recruitment.148.1.csv")  |\textgreater{} mutate(year = 2002, .before = 1)}
\CommentTok{\# PISCO\_data\_2003 \textless{}{-} read\_csv("Datasets/doi\_10.6085\_AA\_pisco\_recruitment.150.1.data") |\textgreater{} mutate(year = 2003, .before = 1)}
\CommentTok{\# PISCO\_data\_2004 \textless{}{-} read\_csv("Datasets/doi\_10.6085\_AA\_pisco\_recruitment.152.1.csv")  |\textgreater{} mutate(year = 2004, .before = 1)}
\CommentTok{\# PISCO\_data\_2005 \textless{}{-} read\_csv("Datasets/doi\_10.6085\_AA\_pisco\_recruitment.154.1.csv")  |\textgreater{} mutate(year = 2005, .before = 1)}
\CommentTok{\# PISCO\_data\_2006 \textless{}{-} read\_csv("Datasets/doi\_10.6085\_AA\_pisco\_recruitment.156.1.data") |\textgreater{} mutate(year = 2006, .before = 1)}
\CommentTok{\# PISCO\_data\_2007 \textless{}{-} read\_csv("Datasets/doi\_10.6085\_AA\_pisco\_recruitment.1364.1.data")|\textgreater{} mutate(year = 2007, .before = 1)}
\CommentTok{\# PISCO\_data\_2008 \textless{}{-} read\_csv("Datasets/doi\_10.6085\_AA\_pisco\_recruitment.1366.1.csv") |\textgreater{} mutate(year = 2008, .before = 1)}
\CommentTok{\# PISCO\_data\_2009 \textless{}{-} read\_csv("Datasets/doi\_10.6085\_AA\_pisco\_recruitment.1438.1.csv") |\textgreater{} mutate(year = 2009, .before = 1)}
\CommentTok{\# PISCO\_data\_2010 \textless{}{-} read\_csv("Datasets/doi\_10.6085\_AA\_pisco\_recruitment.1446.1.csv") |\textgreater{} mutate(year = 2010, .before = 1)}
\CommentTok{\# PISCO\_data\_2011 \textless{}{-} read\_csv("Datasets/doi\_10.6085\_AA\_pisco\_recruitment.1464.1.txt") |\textgreater{} mutate(year = 2011, .before = 1)}
\CommentTok{\# }
\CommentTok{\# \# Combine all datasets}
\CommentTok{\# PISCO\_all\_years \textless{}{-} bind\_rows(PISCO\_data\_1989, PISCO\_data\_1990, PISCO\_data\_1991, }
\CommentTok{\#                              PISCO\_data\_1992, PISCO\_data\_1993, PISCO\_data\_1994, }
\CommentTok{\#                              PISCO\_data\_1995, PISCO\_data\_1996, PISCO\_data\_1998,}
\CommentTok{\#                              PISCO\_data\_1999, PISCO\_data\_2000, PISCO\_data\_2001,}
\CommentTok{\#                              PISCO\_data\_2002, PISCO\_data\_2003, PISCO\_data\_2004, }
\CommentTok{\#                              PISCO\_data\_2005, PISCO\_data\_2006, PISCO\_data\_2007, }
\CommentTok{\#                              PISCO\_data\_2008, PISCO\_data\_2009, PISCO\_data\_2010, }
\CommentTok{\#                              PISCO\_data\_2011)}

\CommentTok{\# \# Print and explore}
\CommentTok{\# print()}
\CommentTok{\# summary()}
 
\CommentTok{\# Save table}
\CommentTok{\# write.csv(PISCO\_all\_years, "PISCO\_all\_years.csv", row.names = FALSE)}
\end{Highlighting}
\end{Shaded}

\#\#INPUT DATA

\begin{Shaded}
\begin{Highlighting}[]
\CommentTok{\# Load PISCO data}
\NormalTok{pisco }\OtherTok{\textless{}{-}} \FunctionTok{read\_csv}\NormalTok{(}\StringTok{"PISCO\_all\_years.csv"}\NormalTok{)}
\end{Highlighting}
\end{Shaded}

\begin{verbatim}
## Rows: 77630 Columns: 15
## -- Column specification --------------------------------------------------------
## Delimiter: ","
## chr  (7): sample_month, site_code, exposure, zone, collector_type, sampler, ...
## dbl  (6): year, replicate, proportion_sampled, count_classcode, count, metho...
## date (2): deploy_date, collect_date
## 
## i Use `spec()` to retrieve the full column specification for this data.
## i Specify the column types or set `show_col_types = FALSE` to quiet this message.
\end{verbatim}

\begin{Shaded}
\begin{Highlighting}[]
\CommentTok{\# Filter for urchin taxa and remove unwanted sites}
\NormalTok{urchins\_raw }\OtherTok{\textless{}{-}}\NormalTok{ pisco }\SpecialCharTok{\%\textgreater{}\%}
  \FunctionTok{filter}\NormalTok{(count\_classcode }\SpecialCharTok{\%in\%} \FunctionTok{c}\NormalTok{(}\StringTok{"66"}\NormalTok{, }\StringTok{"67"}\NormalTok{, }\StringTok{"70"}\NormalTok{),}
         \SpecialCharTok{!}\NormalTok{site\_code }\SpecialCharTok{\%in\%} \FunctionTok{c}\NormalTok{(}\StringTok{"CMEN00"}\NormalTok{, }\StringTok{"CMES00"}\NormalTok{, }\StringTok{"IFBRXX"}\NormalTok{, }\StringTok{"KHLX00"}\NormalTok{,}
                           \StringTok{"PSGX00"}\NormalTok{, }\StringTok{"SHCX00"}\NormalTok{, }\StringTok{"TRHX00"}\NormalTok{, }\StringTok{"IPTAXX"}\NormalTok{))}

\CommentTok{\# Aggregate counts per transect, then per year}
\NormalTok{urchins }\OtherTok{\textless{}{-}}\NormalTok{ urchins\_raw }\SpecialCharTok{\%\textgreater{}\%}
  \FunctionTok{group\_by}\NormalTok{(year, site\_code, zone, replicate) }\SpecialCharTok{\%\textgreater{}\%}
  \FunctionTok{summarise}\NormalTok{(}\AttributeTok{count =} \FunctionTok{sum}\NormalTok{(count), }\AttributeTok{.groups =} \StringTok{"drop"}\NormalTok{) }\SpecialCharTok{\%\textgreater{}\%}
  \FunctionTok{mutate}\NormalTok{(}\AttributeTok{log\_count =} \FunctionTok{log}\NormalTok{(count }\SpecialCharTok{+} \DecValTok{1}\NormalTok{))  }\CommentTok{\# Log{-}transform counts}

\CommentTok{\# Quick visual check}
\FunctionTok{hist}\NormalTok{(urchins}\SpecialCharTok{$}\NormalTok{count)}
\end{Highlighting}
\end{Shaded}

\includegraphics{PISCO_urchin_tuffy_files/figure-latex/unnamed-chunk-2-1.pdf}

\begin{Shaded}
\begin{Highlighting}[]
\FunctionTok{hist}\NormalTok{(urchins}\SpecialCharTok{$}\NormalTok{log\_count)}
\end{Highlighting}
\end{Shaded}

\includegraphics{PISCO_urchin_tuffy_files/figure-latex/unnamed-chunk-2-2.pdf}

\subsection{MODEL URCHIN RECRUITMENT}\label{model-urchin-recruitment}

Objective: 1. estimate the normalized year to year variance (SD) by site
of incoming recruits Note: sea urchin recruitment follows a non-normal
distribution

Aim: I'd be building a hierarchical structure where my incoming settlers
data informs a non-normal distributed annual means, and those annual
means inform a normally distributed year to year variance (SD).

Model(s) method(s): 1. BRMS 2. NIMBLE

\subsection{1) BRMS MODEL}\label{brms-model}

\begin{Shaded}
\begin{Highlighting}[]
\CommentTok{\# Load libraries}
\FunctionTok{library}\NormalTok{(brms)}
\end{Highlighting}
\end{Shaded}

\begin{verbatim}
## Warning: package 'brms' was built under R version 4.4.3
\end{verbatim}

\begin{verbatim}
## Loading required package: Rcpp
\end{verbatim}

\begin{verbatim}
## Loading 'brms' package (version 2.22.0). Useful instructions
## can be found by typing help('brms'). A more detailed introduction
## to the package is available through vignette('brms_overview').
\end{verbatim}

\begin{verbatim}
## 
## Attaching package: 'brms'
\end{verbatim}

\begin{verbatim}
## The following object is masked from 'package:stats':
## 
##     ar
\end{verbatim}

\begin{Shaded}
\begin{Highlighting}[]
\FunctionTok{library}\NormalTok{(posterior)}
\end{Highlighting}
\end{Shaded}

\begin{verbatim}
## Warning: package 'posterior' was built under R version 4.4.3
\end{verbatim}

\begin{verbatim}
## This is posterior version 1.6.1
\end{verbatim}

\begin{verbatim}
## 
## Attaching package: 'posterior'
\end{verbatim}

\begin{verbatim}
## The following objects are masked from 'package:stats':
## 
##     mad, sd, var
\end{verbatim}

\begin{verbatim}
## The following objects are masked from 'package:base':
## 
##     %in%, match
\end{verbatim}

\begin{Shaded}
\begin{Highlighting}[]
\CommentTok{\# Fit zero{-}inflated Poisson model with nested random effects}
\NormalTok{brms\_model }\OtherTok{\textless{}{-}} \FunctionTok{brm}\NormalTok{(}
  \AttributeTok{formula =}\NormalTok{ count }\SpecialCharTok{\textasciitilde{}}\NormalTok{ site\_code }\SpecialCharTok{+}\NormalTok{ (}\DecValTok{1} \SpecialCharTok{|}\NormalTok{ site\_code }\SpecialCharTok{/}\NormalTok{ year), }\CommentTok{\#Nested (site‐specific) Year Variance}
  \CommentTok{\#formula = count \textasciitilde{} site\_code + (1 | site\_code) + (1 | year), \#Global (pooled) Year Variance}
  \AttributeTok{family  =} \FunctionTok{zero\_inflated\_poisson}\NormalTok{(),}
  \AttributeTok{data    =}\NormalTok{ urchins,}
  \AttributeTok{iter    =} \DecValTok{4000}\NormalTok{,}
  \AttributeTok{warmup  =} \DecValTok{1000}\NormalTok{,}
  \AttributeTok{thin    =} \DecValTok{3}\NormalTok{,}
  \AttributeTok{chains  =} \DecValTok{3}\NormalTok{,}
  \AttributeTok{cores   =} \DecValTok{3}\NormalTok{)}
\end{Highlighting}
\end{Shaded}

\begin{verbatim}
## Compiling Stan program...
\end{verbatim}

\begin{verbatim}
## WARNING: Rtools is required to build R packages, but is not currently installed.
## 
## Please download and install the appropriate version of Rtools for 4.4.1 from
## https://cran.r-project.org/bin/windows/Rtools/.
\end{verbatim}

\begin{verbatim}
## Trying to compile a simple C file
\end{verbatim}

\begin{verbatim}
## Running "C:/PROGRA~1/R/R-44~1.1/bin/x64/Rcmd.exe" SHLIB foo.c
## using C compiler: 'gcc.exe (GCC) 13.3.0'
## gcc  -I"C:/PROGRA~1/R/R-44~1.1/include" -DNDEBUG   -I"C:/Users/pinosa/AppData/Local/R/win-library/4.4/Rcpp/include/"  -I"C:/Users/pinosa/AppData/Local/R/win-library/4.4/RcppEigen/include/"  -I"C:/Users/pinosa/AppData/Local/R/win-library/4.4/RcppEigen/include/unsupported"  -I"C:/Users/pinosa/AppData/Local/R/win-library/4.4/BH/include" -I"C:/Users/pinosa/AppData/Local/R/win-library/4.4/StanHeaders/include/src/"  -I"C:/Users/pinosa/AppData/Local/R/win-library/4.4/StanHeaders/include/"  -I"C:/Users/pinosa/AppData/Local/R/win-library/4.4/RcppParallel/include/" -DRCPP_PARALLEL_USE_TBB=1 -I"C:/Users/pinosa/AppData/Local/R/win-library/4.4/rstan/include" -DEIGEN_NO_DEBUG  -DBOOST_DISABLE_ASSERTS  -DBOOST_PENDING_INTEGER_LOG2_HPP  -DSTAN_THREADS  -DUSE_STANC3 -DSTRICT_R_HEADERS  -DBOOST_PHOENIX_NO_VARIADIC_EXPRESSION  -D_HAS_AUTO_PTR_ETC=0  -include "C:/Users/pinosa/AppData/Local/R/win-library/4.4/StanHeaders/include/stan/math/prim/fun/Eigen.hpp"  -std=c++1y    -I"C:/rtools44/x86_64-w64-mingw32.static.posix/include"     -O2 -Wall  -mfpmath=sse -msse2 -mstackrealign  -c foo.c -o foo.o
## cc1.exe: warning: command-line option '-std=c++14' is valid for C++/ObjC++ but not for C
## In file included from C:/Users/pinosa/AppData/Local/R/win-library/4.4/RcppEigen/include/Eigen/Core:19,
##                  from C:/Users/pinosa/AppData/Local/R/win-library/4.4/RcppEigen/include/Eigen/Dense:1,
##                  from C:/Users/pinosa/AppData/Local/R/win-library/4.4/StanHeaders/include/stan/math/prim/fun/Eigen.hpp:22,
##                  from <command-line>:
## C:/Users/pinosa/AppData/Local/R/win-library/4.4/RcppEigen/include/Eigen/src/Core/util/Macros.h:679:10: fatal error: cmath: No such file or directory
##   679 | #include <cmath>
##       |          ^~~~~~~
## compilation terminated.
## make: *** [C:/PROGRA~1/R/R-44~1.1/etc/x64/Makeconf:289: foo.o] Error 1
\end{verbatim}

\begin{verbatim}
## WARNING: Rtools is required to build R packages, but is not currently installed.
## 
## Please download and install the appropriate version of Rtools for 4.4.1 from
## https://cran.r-project.org/bin/windows/Rtools/.
\end{verbatim}

\begin{verbatim}
## Start sampling
\end{verbatim}

\begin{verbatim}
## Warning: There were 241 divergent transitions after warmup. See
## https://mc-stan.org/misc/warnings.html#divergent-transitions-after-warmup
## to find out why this is a problem and how to eliminate them.
\end{verbatim}

\begin{verbatim}
## Warning: Examine the pairs() plot to diagnose sampling problems
\end{verbatim}

\begin{verbatim}
## Warning: Tail Effective Samples Size (ESS) is too low, indicating posterior variances and tail quantiles may be unreliable.
## Running the chains for more iterations may help. See
## https://mc-stan.org/misc/warnings.html#tail-ess
\end{verbatim}

\begin{Shaded}
\begin{Highlighting}[]
\CommentTok{\# Model summary and diagnostics}
\FunctionTok{summary}\NormalTok{(brms\_model)}
\end{Highlighting}
\end{Shaded}

\begin{verbatim}
## Warning: There were 241 divergent transitions after warmup. Increasing
## adapt_delta above 0.8 may help. See
## http://mc-stan.org/misc/warnings.html#divergent-transitions-after-warmup
\end{verbatim}

\begin{verbatim}
##  Family: zero_inflated_poisson 
##   Links: mu = log; zi = identity 
## Formula: count ~ site_code + (1 | site_code/year) 
##    Data: urchins (Number of observations: 534) 
##   Draws: 3 chains, each with iter = 4000; warmup = 1000; thin = 3;
##          total post-warmup draws = 3000
## 
## Multilevel Hyperparameters:
## ~site_code (Number of levels: 14) 
##               Estimate Est.Error l-95% CI u-95% CI Rhat Bulk_ESS Tail_ESS
## sd(Intercept)     1.85      1.41     0.07     5.17 1.01      473      211
## 
## ~site_code:year (Number of levels: 104) 
##               Estimate Est.Error l-95% CI u-95% CI Rhat Bulk_ESS Tail_ESS
## sd(Intercept)     1.97      0.34     1.41     2.75 1.01      955     2370
## 
## Regression Coefficients:
##                 Estimate Est.Error l-95% CI u-95% CI Rhat Bulk_ESS Tail_ESS
## Intercept          -1.65      2.11    -6.11     2.98 1.00      549      943
## site_codeCARX00     0.06      3.23    -6.30     8.05 1.01      465      322
## site_codeCBLN00     0.12      3.30    -7.20     7.18 1.00      813     1166
## site_codeCBLS00    -2.35      4.09   -10.25     6.78 1.01      328      169
## site_codeCBLX00    -0.43      3.21    -7.24     6.04 1.01     1102     1217
## site_codeCMRX00    -0.81      3.21    -7.61     5.86 1.00      753     1624
## site_codeFCKX00    -0.72      3.06    -7.80     5.39 1.00      713     1154
## site_codeGHVX00     0.39      3.11    -6.22     6.81 1.00      807     1138
## site_codeMBYX00    -0.48      3.32    -7.56     6.76 1.00      965     1398
## site_codePOHX00    -3.38      3.52   -10.99     3.86 1.01      498      230
## site_codeRKPX00    -1.05      3.33    -8.14     6.44 1.00      546      159
## site_codeSHLX00     0.35      2.93    -5.97     6.85 1.01      786      983
## site_codeSRKX00    -0.73      3.23    -8.08     5.42 1.01      663      676
## site_codeYBHX00    -0.71      3.15    -7.90     6.38 1.01      495      335
## 
## Further Distributional Parameters:
##    Estimate Est.Error l-95% CI u-95% CI Rhat Bulk_ESS Tail_ESS
## zi     0.24      0.07     0.11     0.37 1.00     2477     2379
## 
## Draws were sampled using sampling(NUTS). For each parameter, Bulk_ESS
## and Tail_ESS are effective sample size measures, and Rhat is the potential
## scale reduction factor on split chains (at convergence, Rhat = 1).
\end{verbatim}

\begin{Shaded}
\begin{Highlighting}[]
\FunctionTok{plot}\NormalTok{(brms\_model)}
\end{Highlighting}
\end{Shaded}

\includegraphics{PISCO_urchin_tuffy_files/figure-latex/unnamed-chunk-3-1.pdf}
\includegraphics{PISCO_urchin_tuffy_files/figure-latex/unnamed-chunk-3-2.pdf}
\includegraphics{PISCO_urchin_tuffy_files/figure-latex/unnamed-chunk-3-3.pdf}
\includegraphics{PISCO_urchin_tuffy_files/figure-latex/unnamed-chunk-3-4.pdf}

\begin{Shaded}
\begin{Highlighting}[]
\CommentTok{\# Posterior draws for yearly variation across sites}
\NormalTok{post }\OtherTok{\textless{}{-}} \FunctionTok{as\_draws\_df}\NormalTok{(brms\_model, }\AttributeTok{transform =} \ConstantTok{TRUE}\NormalTok{) }\CommentTok{\#FALSE for log scale}
\FunctionTok{mean}\NormalTok{(post[[}\StringTok{"sd\_site\_code:year\_\_Intercept"}\NormalTok{]]) }\CommentTok{\#number we want}
\end{Highlighting}
\end{Shaded}

\begin{verbatim}
## [1] 1.972638
\end{verbatim}

\subsection{2) NIMBLE VERSION OF THE
MODEL}\label{nimble-version-of-the-model}

\begin{Shaded}
\begin{Highlighting}[]
\CommentTok{\# Load libraries}
\FunctionTok{library}\NormalTok{(nimble)}
\end{Highlighting}
\end{Shaded}

\begin{verbatim}
## Warning: package 'nimble' was built under R version 4.4.3
\end{verbatim}

\begin{verbatim}
## nimble version 1.3.0 is loaded.
## For more information on NIMBLE and a User Manual,
## please visit https://R-nimble.org.
## 
## Note for advanced users who have written their own MCMC samplers:
##   As of version 0.13.0, NIMBLE's protocol for handling posterior
##   predictive nodes has changed in a way that could affect user-defined
##   samplers in some situations. Please see Section 15.5.1 of the User Manual.
\end{verbatim}

\begin{verbatim}
## 
## Attaching package: 'nimble'
\end{verbatim}

\begin{verbatim}
## The following object is masked from 'package:stats':
## 
##     simulate
\end{verbatim}

\begin{verbatim}
## The following object is masked from 'package:base':
## 
##     declare
\end{verbatim}

\begin{Shaded}
\begin{Highlighting}[]
\FunctionTok{source}\NormalTok{(}\StringTok{"attach.nimble.R"}\NormalTok{)}

\CommentTok{\# Define model}
\NormalTok{model\_code }\OtherTok{\textless{}{-}} \FunctionTok{nimbleCode}\NormalTok{(\{}
  
  \CommentTok{\# Hyper priors}
\NormalTok{  beta0      }\SpecialCharTok{\textasciitilde{}} \FunctionTok{dnorm}\NormalTok{(}\DecValTok{0}\NormalTok{, }\AttributeTok{sd =} \DecValTok{10}\NormalTok{)}
\NormalTok{  sigma\_site }\SpecialCharTok{\textasciitilde{}} \FunctionTok{dunif}\NormalTok{(}\DecValTok{0}\NormalTok{, }\DecValTok{5}\NormalTok{)}
\NormalTok{  sigma\_sy   }\SpecialCharTok{\textasciitilde{}} \FunctionTok{dunif}\NormalTok{(}\DecValTok{0}\NormalTok{, }\DecValTok{5}\NormalTok{)}
\NormalTok{  sigma\_obs  }\SpecialCharTok{\textasciitilde{}} \FunctionTok{dunif}\NormalTok{(}\DecValTok{0}\NormalTok{, }\DecValTok{5}\NormalTok{)}
  
  \CommentTok{\# Random intercept for each site:}
  \ControlFlowTok{for}\NormalTok{ (s }\ControlFlowTok{in} \DecValTok{1}\SpecialCharTok{:}\NormalTok{n\_site) \{}
\NormalTok{    b\_site[s] }\SpecialCharTok{\textasciitilde{}} \FunctionTok{dnorm}\NormalTok{(}\DecValTok{0}\NormalTok{, }\AttributeTok{sd =}\NormalTok{ sigma\_site)}
\NormalTok{  \} }\CommentTok{\#s}
  
  \CommentTok{\# Random intercept for each site×year combo:}
  \ControlFlowTok{for}\NormalTok{ (j }\ControlFlowTok{in} \DecValTok{1}\SpecialCharTok{:}\NormalTok{n\_sy) \{}
\NormalTok{    b\_sy[j] }\SpecialCharTok{\textasciitilde{}} \FunctionTok{dnorm}\NormalTok{(}\DecValTok{0}\NormalTok{, }\AttributeTok{sd =}\NormalTok{ sigma\_sy)}
\NormalTok{  \} }\CommentTok{\#j}
  
  \CommentTok{\# “Normalized” (original‐scale) version of sigma\_sy:}
\NormalTok{  sigma\_sy\_n }\OtherTok{\textless{}{-}} \FunctionTok{exp}\NormalTok{(sigma\_sy) }
  
  \CommentTok{\# Likelihood: log(count+1) \textasciitilde{} Normal:}
  \ControlFlowTok{for}\NormalTok{ (i }\ControlFlowTok{in} \DecValTok{1}\SpecialCharTok{:}\NormalTok{N) \{}
\NormalTok{    mu[i]     }\OtherTok{\textless{}{-}}\NormalTok{ beta0 }\SpecialCharTok{+}\NormalTok{ b\_site[site[i]] }\SpecialCharTok{+}\NormalTok{ b\_sy[sy\_index[i]] }\CommentTok{\#process model}
\NormalTok{    log\_y[i] }\SpecialCharTok{\textasciitilde{}} \FunctionTok{dnorm}\NormalTok{(mu[i], }\AttributeTok{sd =}\NormalTok{ sigma\_obs) }\CommentTok{\#observation model}
\NormalTok{  \} }\CommentTok{\#i}
  
\NormalTok{\}) }\CommentTok{\#model\_code}

\CommentTok{\# Prepare input data}
\NormalTok{urchins }\OtherTok{\textless{}{-}}\NormalTok{ urchins }\SpecialCharTok{\%\textgreater{}\%}
  \FunctionTok{mutate}\NormalTok{(}\AttributeTok{site =} \FunctionTok{as.integer}\NormalTok{(}\FunctionTok{factor}\NormalTok{(site\_code)),}
         \AttributeTok{site\_year =} \FunctionTok{paste}\NormalTok{(site\_code, year, }\AttributeTok{sep =} \StringTok{"\_"}\NormalTok{),}
         \AttributeTok{sy\_index =} \FunctionTok{as.integer}\NormalTok{(}\FunctionTok{factor}\NormalTok{(site\_year)))}

\NormalTok{log\_y    }\OtherTok{\textless{}{-}}\NormalTok{ urchins}\SpecialCharTok{$}\NormalTok{log\_count}
\NormalTok{site     }\OtherTok{\textless{}{-}}\NormalTok{ urchins}\SpecialCharTok{$}\NormalTok{site}
\NormalTok{sy\_index }\OtherTok{\textless{}{-}}\NormalTok{ urchins}\SpecialCharTok{$}\NormalTok{sy\_index}
\NormalTok{N        }\OtherTok{\textless{}{-}} \FunctionTok{nrow}\NormalTok{(urchins)}
\NormalTok{n\_site   }\OtherTok{\textless{}{-}} \FunctionTok{length}\NormalTok{(}\FunctionTok{unique}\NormalTok{(site))}
\NormalTok{n\_sy     }\OtherTok{\textless{}{-}} \FunctionTok{length}\NormalTok{(}\FunctionTok{unique}\NormalTok{(sy\_index))}

\NormalTok{nimble.data }\OtherTok{\textless{}{-}} \FunctionTok{list}\NormalTok{(}\AttributeTok{log\_y =}\NormalTok{ log\_y, }
                    \AttributeTok{site =}\NormalTok{ site, }
                    \AttributeTok{sy\_index =}\NormalTok{ sy\_index)}

\NormalTok{nimble.constants }\OtherTok{\textless{}{-}} \FunctionTok{list}\NormalTok{(}\AttributeTok{N =}\NormalTok{ N, }
                         \AttributeTok{n\_site =}\NormalTok{ n\_site, }
                         \AttributeTok{n\_sy =}\NormalTok{ n\_sy)}

\CommentTok{\# Initial values}
\NormalTok{inits }\OtherTok{\textless{}{-}} \FunctionTok{list}\NormalTok{(}\AttributeTok{beta0 =} \DecValTok{0}\NormalTok{,}
              \AttributeTok{b\_site =} \FunctionTok{rep}\NormalTok{(}\DecValTok{0}\NormalTok{, n\_site),}
              \AttributeTok{b\_sy =} \FunctionTok{rep}\NormalTok{(}\DecValTok{0}\NormalTok{, n\_sy),}
              \AttributeTok{sigma\_site =} \DecValTok{1}\NormalTok{,}
              \AttributeTok{sigma\_sy =} \DecValTok{1}\NormalTok{,}
              \AttributeTok{sigma\_obs =} \DecValTok{1}\NormalTok{)}

\CommentTok{\# Parameters}
\NormalTok{parameters }\OtherTok{\textless{}{-}} \FunctionTok{c}\NormalTok{(}\StringTok{"beta0"}\NormalTok{, }\StringTok{"sigma\_site"}\NormalTok{, }\StringTok{"sigma\_sy"}\NormalTok{, }\StringTok{"sigma\_sy\_n"}\NormalTok{, }\StringTok{"sigma\_obs"}\NormalTok{)}

\CommentTok{\# Run MCMC}
\NormalTok{mcmc.output }\OtherTok{\textless{}{-}} \FunctionTok{nimbleMCMC}\NormalTok{(}\AttributeTok{code      =}\NormalTok{ model\_code, }
                          \AttributeTok{data      =}\NormalTok{ nimble.data,}
                          \AttributeTok{constants =}\NormalTok{ nimble.constants, }
                          \AttributeTok{inits     =}\NormalTok{ inits,}
                          \AttributeTok{monitors  =}\NormalTok{ parameters,}
                          \AttributeTok{niter     =} \DecValTok{40000}\NormalTok{, }
                          \AttributeTok{nburnin   =} \DecValTok{20000}\NormalTok{, }
                          \AttributeTok{thin      =} \DecValTok{40}\NormalTok{, }
                          \AttributeTok{nchains   =} \DecValTok{3}\NormalTok{,}
                          \AttributeTok{summary   =} \ConstantTok{TRUE}\NormalTok{,}
                          \AttributeTok{samplesAsCoda =} \ConstantTok{TRUE}\NormalTok{)}
\end{Highlighting}
\end{Shaded}

\begin{verbatim}
## Defining model
\end{verbatim}

\begin{verbatim}
##   [Note] Detected use of non-constant indexes: site[1], site[2], site[3], site[4], site[5], site[6], site[7], site[8], site[9], site[10], site[11], site[12], site[13], site[14], site[15], site[16], site[17], site[18], site[19], site[20], site[21], site[22], site[23], site[24], site[25], site[26], site[27], site[28], site[29], site[30], site[31], site[32], site[33], site[34], site[35], site[36], site[37], site[38], site[39], site[40], site[41], site[42], site[43], site[44], site[45], site[46], site[47], site[48], site[49], site[50], sy_index[1], sy_index[2], sy_index[3], sy_index[4], sy_index[5], sy_index[6], sy_index[7], sy_index[8], sy_index[9], sy_index[10], sy_index[11], sy_index[12], sy_index[13], sy_index[14], sy_index[15], sy_index[16], sy_index[17], sy_index[18], sy_index[19], sy_index[20], sy_index[21], sy_index[22], sy_index[23], sy_index[24], sy_index[25], sy_index[26], sy_index[27], sy_index[28], sy_index[29], sy_index[30], sy_index[31], sy_index[32], sy_index[33], sy_index[34], sy_index[35], sy_index[36], sy_index[37], sy_index[38], sy_index[39], sy_index[40], sy_index[41], sy_index[42], sy_index[43], sy_index[44], sy_index[45], sy_index[46], sy_index[47], sy_index[48], sy_index[49], sy_index[50], ...
##          For computational efficiency we recommend specifying these in 'constants'.
\end{verbatim}

\begin{verbatim}
## Building model
\end{verbatim}

\begin{verbatim}
## Setting data and initial values
\end{verbatim}

\begin{verbatim}
## Running calculate on model
##   [Note] Any error reports that follow may simply reflect missing values in model variables.
\end{verbatim}

\begin{verbatim}
## Checking model sizes and dimensions
\end{verbatim}

\begin{verbatim}
## Checking model calculations
\end{verbatim}

\begin{verbatim}
## Compiling
##   [Note] This may take a minute.
##   [Note] Use 'showCompilerOutput = TRUE' to see C++ compilation details.
\end{verbatim}

\begin{verbatim}
## running chain 1...
\end{verbatim}

\begin{verbatim}
## |-------------|-------------|-------------|-------------|
## |-------------------------------------------------------|
\end{verbatim}

\begin{verbatim}
## running chain 2...
\end{verbatim}

\begin{verbatim}
## |-------------|-------------|-------------|-------------|
## |-------------------------------------------------------|
\end{verbatim}

\begin{verbatim}
## running chain 3...
\end{verbatim}

\begin{verbatim}
## |-------------|-------------|-------------|-------------|
## |-------------------------------------------------------|
\end{verbatim}

\begin{Shaded}
\begin{Highlighting}[]
\CommentTok{\# Summarize posterior}
\FunctionTok{attach.nimble}\NormalTok{(mcmc.output}\SpecialCharTok{$}\NormalTok{samples)}
\end{Highlighting}
\end{Shaded}

\begin{verbatim}
## 
## Attaching package: 'data.table'
\end{verbatim}

\begin{verbatim}
## The following object is masked from 'package:nimble':
## 
##     cube
\end{verbatim}

\begin{verbatim}
## The following objects are masked from 'package:lubridate':
## 
##     hour, isoweek, mday, minute, month, quarter, second, wday, week,
##     yday, year
\end{verbatim}

\begin{verbatim}
## The following object is masked from 'package:purrr':
## 
##     transpose
\end{verbatim}

\begin{verbatim}
## The following objects are masked from 'package:dplyr':
## 
##     between, first, last
\end{verbatim}

\begin{Shaded}
\begin{Highlighting}[]
\FunctionTok{summary}\NormalTok{(mcmc.output}\SpecialCharTok{$}\NormalTok{samples)}
\end{Highlighting}
\end{Shaded}

\begin{verbatim}
## 
## Iterations = 1:500
## Thinning interval = 1 
## Number of chains = 3 
## Sample size per chain = 500 
## 
## 1. Empirical mean and standard deviation for each variable,
##    plus standard error of the mean:
## 
##               Mean      SD  Naive SE Time-series SE
## beta0      0.20816 0.03761 0.0009711      0.0010079
## sigma_obs  0.31699 0.01096 0.0002830      0.0002905
## sigma_site 0.05734 0.04051 0.0010459      0.0020329
## sigma_sy   0.26541 0.02699 0.0006968      0.0006967
## sigma_sy_n 1.30444 0.03536 0.0009130      0.0009129
## 
## 2. Quantiles for each variable:
## 
##                2.5%    25%    50%     75%  97.5%
## beta0      0.132903 0.1842 0.2096 0.23304 0.2808
## sigma_obs  0.296473 0.3092 0.3168 0.32413 0.3389
## sigma_site 0.004157 0.0255 0.0503 0.08092 0.1535
## sigma_sy   0.216749 0.2479 0.2638 0.28250 0.3206
## sigma_sy_n 1.242032 1.2814 1.3018 1.32644 1.3779
\end{verbatim}

\begin{Shaded}
\begin{Highlighting}[]
\CommentTok{\# Extract yearly variation across sites}
\FunctionTok{quantile}\NormalTok{(sigma\_sy\_n, }\FunctionTok{c}\NormalTok{(}\FloatTok{0.025}\NormalTok{, }\FloatTok{0.5}\NormalTok{, }\FloatTok{0.975}\NormalTok{))}
\end{Highlighting}
\end{Shaded}

\begin{verbatim}
##     2.5%      50%    97.5% 
## 1.242032 1.301850 1.377948
\end{verbatim}

\begin{Shaded}
\begin{Highlighting}[]
\FunctionTok{mean}\NormalTok{(sigma\_sy\_n) }\CommentTok{\#Number we want}
\end{Highlighting}
\end{Shaded}

\begin{verbatim}
## [1] 1.304436
\end{verbatim}

\begin{Shaded}
\begin{Highlighting}[]
\FunctionTok{hist}\NormalTok{(sigma\_sy\_n, }\AttributeTok{main =} \StringTok{"Posterior of Normalized Year{-}to{-}Year SD"}\NormalTok{,}
     \AttributeTok{xlab =} \StringTok{"Normalized SD"}\NormalTok{)}
\end{Highlighting}
\end{Shaded}

\includegraphics{PISCO_urchin_tuffy_files/figure-latex/unnamed-chunk-4-1.pdf}

\begin{Shaded}
\begin{Highlighting}[]
\FunctionTok{hist}\NormalTok{(sigma\_sy, }\AttributeTok{main =} \StringTok{"Posterior of Non{-}Normal Year{-}to{-}Year SD"}\NormalTok{,}
     \AttributeTok{xlab =} \StringTok{"Non{-}Normal SD"}\NormalTok{)}
\end{Highlighting}
\end{Shaded}

\includegraphics{PISCO_urchin_tuffy_files/figure-latex/unnamed-chunk-4-2.pdf}

\begin{Shaded}
\begin{Highlighting}[]
\CommentTok{\# \# Optional diagnostics}
\FunctionTok{MCMCtrace}\NormalTok{(}\AttributeTok{object =}\NormalTok{ mcmc.output}\SpecialCharTok{$}\NormalTok{samples,}
          \AttributeTok{pdf =} \ConstantTok{FALSE}\NormalTok{,}
          \AttributeTok{ind =} \ConstantTok{TRUE}\NormalTok{,}
          \CommentTok{\# params = "sigma\_sy\_n")}
          \AttributeTok{params =} \FunctionTok{c}\NormalTok{(}\StringTok{"beta0"}\NormalTok{, }\StringTok{"sigma\_site"}\NormalTok{, }\StringTok{"sigma\_sy"}\NormalTok{, }\StringTok{"sigma\_sy\_n"}\NormalTok{, }\StringTok{"sigma\_obs"}\NormalTok{)) }\CommentTok{\#Posterior predictive check}
\end{Highlighting}
\end{Shaded}

\includegraphics{PISCO_urchin_tuffy_files/figure-latex/unnamed-chunk-4-3.pdf}
\includegraphics{PISCO_urchin_tuffy_files/figure-latex/unnamed-chunk-4-4.pdf}

\begin{Shaded}
\begin{Highlighting}[]
\FunctionTok{gelman.diag}\NormalTok{(mcmc.output}\SpecialCharTok{$}\NormalTok{samples)}
\end{Highlighting}
\end{Shaded}

\begin{verbatim}
## Potential scale reduction factors:
## 
##            Point est. Upper C.I.
## beta0               1       1.01
## sigma_obs           1       1.00
## sigma_site          1       1.01
## sigma_sy            1       1.00
## sigma_sy_n          1       1.00
## 
## Multivariate psrf
## 
## 1
\end{verbatim}

\begin{Shaded}
\begin{Highlighting}[]
\CommentTok{\# mcmcplot(mcmc.output$samples)}
\end{Highlighting}
\end{Shaded}

\subsection{OLD VERSIONS WITH THE WRONG APPROACH
(ignore)}\label{old-versions-with-the-wrong-approach-ignore}

\end{document}
